%-------------------------------------------------------------------------------
%  PACKAGES AND THEMES
%-------------------------------------------------------------------------------
\documentclass{beamer}
\mode<presentation> {
  %-----------------------------------------------------------------------------
  % Назначение всех переменных. 
  % P.S. Для работы с подсветкой кода frame должен быть [fragile] !!!!
  %-----------------------------------------------------------------------------
  \usetheme{Szeged}                                  % Красивая тема
  \usecolortheme[]{beaver}
  \setbeamertemplate{headline}{}                     % Выключить шапку
  \setbeamertemplate{footline}[page number]          % Нумерация страниц
  \setbeamertemplate{navigation symbols}{}           % Выключить стрелки
  \usepackage[T2A]{fontenc}                          % Русский язык
  \usepackage[utf8]{inputenc}                        % И кодировка тоже
  \usepackage[english, russian]{babel}               % Языки. Последний главный.
  \usepackage{graphicx}                              % Включаем картинки
  \usepackage{booktabs}                              % Используем \toprule, 
                                                     % \midrule, \bottomrule
  \usepackage{minted}                                % Подсветка кода
  \usepackage{xcolor}                                % Цвета (расширенная)
  \usemintedstyle{monokai}                           % Белый текст для темы
  \definecolor{dark}{HTML}{282828}                   % Всем темную тему!
  \newcommand{\hlt}[1]{\textcolor{red}{\textbf{#1}}} % Подсветка красным ж.
  \newcommand{\fns}{\footnotesize}

  \newcommand{\imagebox}[6]{
      #1
      \begin{figure}[!ht]
          #2
          \ifx&#3&%
          % #3 is empty
          \includegraphics[width=\textwidth,height=\textheight,
                          keepaspectratio]{#4}
          \else
          \includegraphics[scale=#3]{#4}
          \fi
          \ifx&#5&%
          % #5 is empty
          \label{fig:#6}
          \else
          \caption{#5}\label{fig:#6}
          \fi
      \end{figure}
      #1 
      \phantom{imageboxfix}
  }
}
%-------------------------------------------------------------------------------
%  TITLE PAGE
%-------------------------------------------------------------------------------

\title[Язык Go]{Язык Go. Основные возможности и применение} 
\author{Работа \\ студента 411 группы И.~И.~Королёва \\ 
                  и студента 451 группы Д. ~А. ~Емелина}
\institute{{Саратовский государственный университет} \\ им.~Н.~Г.~Чернышевского
            \\[5pt]
            Кафедра математической кибернетики\\ и компьютерных наук\\[5pt]
            Научный руководитель: доцент Иванова ~А.~С. }
\date{2024}

\begin{document}


%-------------------------------------------------------------------------------
%  СЛАЙДЫ
%-------------------------------------------------------------------------------
% Cлайд 01 =====================================================================
\begin{frame} 
  \titlepage 
\end{frame}
% Слайд 02 =====================================================================
\begin{frame}[fragile] \frametitle{Введение}
  Go "--- относительно молодой, но популярный язык программирования.
  \\\\
  Цель, которая стояла перед создателями Go "--- разработать простой и
  эффективный язык программирования, который мог бы использоваться для
  создания качественного программного обеспечения.
  \\\\
  Golang получил третье место в рейтинге языков программирования,
  которые хотели бы освоить разработчики.
\end{frame}
% Слайд 03 =====================================================================
\begin{frame}[fragile] \frametitle{Мотивация}
  \begin{itemize}
    \item Первый процессор Pentium 4 с тактовой частотой 3,7 ГГц был представлен
    еще в 2005 году корпорацией Intel. 
    \item Macbook Pro 2023 имеет тактовую частоту 4 ГГц. 
  \end{itemize}
  Итог "--- почти за два десятилетия мощности не слишком-то изменились.\\

  Для увеличения производительности решили:
  \begin{itemize}
    \item добавлять в процессоры все больше и больше ядер;
    \item ввести гиперпоточность (концепция параллельного выполнения задач);
    \item добавить больше кэша.
  \end{itemize}
\end{frame}
% Слайд 04 =====================================================================
\begin{frame}[fragile] \frametitle{Мотивация}
  Если же нельзя положиться на усовершенствование оборудования, единственный
  выход "--- более эффективное ПО для повышения производительности. 
  \\\\
  По итогу выходит, что разрабатываемое нами ПО и языки программирования должны
  поддерживать параллелизм и быть расширяемыми в условиях постоянного увеличения
  количества ядер. 
  \\\\
  Большинство современных языков программирования (таких как Java, Python)
  поддерживают многопоточность. Но настоящая проблема связана с одновременным
  исполнением, блокировкой потоков, состоянием гонки и взаимоблокировками.
\end{frame}
%-------------------------------------------------------------------------------
% Слайд 05 =====================================================================
\begin{frame}[fragile] \frametitle{Мотивация}
  Возьмем, к примеру, Java. 
  \\\\
  Каждый канал потребляет около 1 Мб объема памяти, и, в конце концов, если вы
  задействуете тысячи потоков, все может закончиться нехваткой памяти. Кроме
  того, взаимодействие между двумя или несколькими потоками – это тоже непросто.
  \\\\
  Язык Go (он же Golang) появился в 2009 году, когда уже были многоядерные
  процессоры. Это позволило не реализовывать параллельные механики поверх уже
  имеющейся спецификации языка, а разработать язык сразу с учетом параллелизма. 
\end{frame}
%-------------------------------------------------------------------------------
% Слайд 06 =====================================================================
\begin{frame}[fragile] \frametitle{Преимущества языка. Простота}
  \textbf{1. Простота} 
  \\\\
  У Go достаточно простой синтаксис (с определенными допущениями), поэтому
  приложения можно разрабатывать быстрее, чем на некоторых других языках.
  \\\\
  Golang достаточно быстро может изучить как полный новичок в
  программировании, так и уже «сформировавшийся программист», тот, кто уже знает
  один или несколько языков.
\end{frame}
%-------------------------------------------------------------------------------
% Слайд 07 =====================================================================
\begin{frame}[fragile] \frametitle{Преимущества языка. Простота}
  Go специально не учитывает многие особенности современных языков ООП в угоду
  простоте кода и удобству его поддержки:
  \begin{enumerate}
    \item Нет классов. Все разделяется на пакеты. Go работает со структурами, а
    не с классами.
    \item Не поддерживает наследование. Это упрощает изменение кода. Без
    наследования Go также становится легко читаемым: нет суперклассов, которые
    следует изучать особо тщательно.
    \item Нет аннотаций (метаданных, прикрепляемых к коду для различных целей).
    \item Нет конструкторов.
    \item Нет исключений.
  \end{enumerate}
\end{frame}
%-------------------------------------------------------------------------------
% Слайд 08 =====================================================================
\begin{frame}[fragile] \frametitle{Преимущества языка. Простота}
  Вышеизложенные изменения делают Golang отличным от других языком, а
  программирование на Go – предельно простым. 
  \\\\
  В отличие от прочих новых языков, таких как Swift, синтаксис Go стабилен. Он
  остался прежним с первого выпуска версии 1.0, что состоялся в 2012 году. Это
  делает его обратно совместимым.
  
\end{frame}
% Слайд 09 =====================================================================
\begin{frame}[fragile] \frametitle{Преимущества языка. Горутины}
  \textbf{2. Горутины}
  \\\\
  У Go есть goroutine вместо потоков. Они потребляют только 2 Кб памяти. Таким
  образом, можно в любой момент активировать миллионы горутин.
  \imagebox{}{\centering}{}{resources/01-goroutines.jpg}{}{pic01}
\end{frame}
% Слайд 10 =====================================================================
\begin{frame}[fragile] \frametitle{Преимущества языка. Горутины}
  \textbf{Горутины} "--- это функции, которые могут работать параллельно, то
  есть программа выполняет несколько строк практически одновременно. Чтобы
  сделать из функции горутину, надо просто написать перед ней ключевое слово 
  \verb|go|.
  \begin{verbatim}
    func server(i int) {
      for {
        print(i)
        time.Sleep(10)
      }
    }
    go server(1)
    go server(2)
  \end{verbatim}
\end{frame}
% Слайд 11 =====================================================================
\begin{frame}[fragile] \frametitle{Преимущества языка. Горутины}
  Результат — практически одновременный вызов, несмотря на задержку
  time.Sleep(10), обеих горутин. Конечно, в небольшой программе это делать
  практически бессмысленно, а вот при вызове множества функций — очень даже
  оправданно. Экономится время, и ресурсы процессора используются равномерно.
  \\\\
  За выполнением горутин в Go следит специальная библиотека времени исполнения:
  она распределяет между ними ядра процессора, может ограничивать число
  доступных ядер. Библиотека помогает запускать огромное количество горутин —
  намного больше, чем позволяет операционная система, — и не требует от
  программиста заниматься распараллеливанием вручную.
\end{frame}
% Слайд 12 =====================================================================
\begin{frame}[fragile] \frametitle{Преимущества языка. Каналы}
  Это что-то вроде общего хранилища данных. Каналы передаются как аргументы
  горутин и помогают им общаться между собой и обмениваться данными. В каналах
  есть очередь и блокировка — чтобы разные горутины не смогли одновременно
  закинуть туда разные данные. Особенность каналов: они позволяют записывать и
  считывать только один тип данных. Например, int — целые числа.
  \begin{verbatim}
    func main() {
        channel := make(chan float32)
    
        fmt.Printf("type of 'c' is %T\n", channel)
        fmt.Printf("value of 'c' is %v\n", channel)
    }
  \end{verbatim}
\end{frame}
% Слайд 13 =====================================================================
\begin{frame}[fragile] \frametitle{Преимущества языка. Горутины}
  Другие преимущества:
  \begin{enumerate}
    \item Горутины используют больше памяти только тогда, когда это необходимо.
    \item Также они запускаются быстрее, чем потоки.
    \item Более того, они идут вместе со встроенными примитивами, чтобы
    безопасно обмениваться данными.
    \item В условиях одновременного использовании структур данных не придется
    прибегать к блокировке мьютексов.
    \item 1 горутина может свободно работать на нескольких потоках. Горутины
    мультиплексируются в небольшое количество потоков ОС.
  \end{enumerate}
\end{frame}
% Слайд 14 =====================================================================
\begin{frame}[fragile] \frametitle{Преимущества языка. Работа с памятью}
  \textbf{3. Работа с памятью и компиляция}
  \\\\
  Одним из наиболее значительных преимуществ языков C и C++ над другими
  современными языками, такими как Java/Python, является их производительность.
  Дело в том, что C и C++ вместо интерпретации компилируются.
  \\\\
  Как и языки низкого уровня, Go является компилируемым. Это означает, что
  производительность почти такая же высокая, как и в низкоуровневых языках. А
  еще он использует сборщик мусора для выделения и удаления объекта.
  
\end{frame}
% Слайд 15 =====================================================================
\begin{frame}[fragile] \frametitle{Преимущества языка. Чистота кода, типизация}
  \textbf{4. Чистота кода и статическая типизация}
  \\\\
  Компилятор Go позволяет держать код «чистым». К примеру, неиспользуемые
  переменные считаются ошибкой компиляции. В Go решается большая часть проблем
  форматирования. Это делается, к примеру, при помощи программы gofmt при
  сохранении или компиляции. Форматирование правится автоматически.
  \\\\
  Язык программирования Go имеет статическую, строгую типизацию. Это означает,
  что типы данных определяются на этапе компиляции и проверяются на соответствие
  во время выполнения программы.
\end{frame}
% Слайд 16 =====================================================================
\begin{frame}[fragile] \frametitle{Преимущества языка. Не нужны фреймворки}
  \textbf{5. Не нужны фреймворки}
  \\\\
  В Go отсутствует традиционное понятие фреймворков, которые часто используются
  в других языках программирования по типу Python, Ruby, JavaScript и т. д.
  Вместо этого в Гоу для создания приложений применяются модули и библиотеки.
  \\\\ 
  Разработчики используют библиотеки и инструменты, чтобы строить мощные и
  надежные Go-приложения, на 100\% подходящие для решения задач, поставленных
  перед ними.
\end{frame}
% Слайд 17 =====================================================================
\begin{frame}[fragile] \frametitle{Преимущества языка. Не нужны фреймворки}
  \textbf{Почему отсутствие фреймворков — это плюс для разработчика?}
  \begin{itemize}
    \item Гибкость.
    \\
    Вместо принудительного использования фреймворков, Go предоставляет
    разработчикам свободу выбора библиотек и инструментов в зависимости от
    каждого проекта. Это позволяет строить приложения, оптимизированные под
    решение конкретных задач.
    \item Модульность. 
    \\
    Программирование на Go — модульный подход к разработке. Этот подход
    позволяет импортировать сторонние библиотеки и использовать только те
    функции, которые действительно нужны в процессе разработчик. То есть
    модульность минимизирует зависимости. 
  \end{itemize}
\end{frame}
% Слайд 18 =====================================================================
\begin{frame}[fragile] \frametitle{Преимущества языка. Не нужны фреймворки}
  \textbf{Почему отсутствие фреймворков — это плюс для разработчика?}
  \begin{itemize}
    \item Множество библиотек.
    \\
    Сообщество Golang активно разрабатывает и поддерживает много полезных
    библиотек и инструментов, которые можно использовать в проектах без
    необходимости использования фреймворков. Эти библиотеки позволяют
    реализовать функциональные возможности по типу маршрутизации HTTP, работы с
    БД, обработки форм и т. п.
    
    \item Производительность. 
    \\
    Go изначально разрабатывался с акцентом на высокую скорость работы.
    Отсутствие больших и тяжеловесных фреймворков способствует более низкому
    потреблению ресурсов рабочей машины и обеспечивает более быстрый запуск
    приложений.
  \end{itemize}
\end{frame}
% Слайд 19 =====================================================================
\begin{frame}[fragile] \frametitle{Преимущества языка. Маппинги}
  \textbf{6. Маппинги}
  \\\\
  Маппинги (Maps) в языке программирования Go представляют собой встроенную
  структуру данных, которая представляет собой набор пар ключ-значение, где
  каждый ключ уникален в рамках маппинга. Они известны также как ассоциативные
  массивы или словари в других языках программирования.
  
  \imagebox{}{\centering}{}{resources/02-mappings.jpg}{}{pic02}
\end{frame}
% Слайд 20 =====================================================================
\begin{frame}[fragile] \frametitle{Преимущества языка. Маппинги}
  \textbf{Основные характеристики маппингов в Go}
  \\\\
  \begin{enumerate}
    \item Уникальные ключи: каждый ключ в маппинге уникален, что означает, что в
    маппинге не может быть двух элементов с одинаковыми ключами.
    \item Неупорядоченность: элементы в маппинге не хранятся в определенном
    порядке. Это означает, что порядок элементов при обходе маппинга не
    гарантирован и может меняться между разными запусками программы.
    \item Динамическое изменение размера: маппинги в Go могут динамически
    увеличиваться или уменьшаться по мере добавления или удаления элементов.
  \end{enumerate}
\end{frame}
% Слайд 21 =====================================================================
\begin{frame}[fragile] \frametitle{Что можно написать на Go?}
  На Go можно написать практически все, за исключением некоторых моментов
  (например, разработки, связанные с машинным обучением — здесь больше подходит
  все же Python с низкоуровневыми оптимизациями на C/C++ и CUDA).
  Основная специализация:
  \begin{itemize}
    \item Веб-приложения: Go имеет нативную поддержку для создания веб-серверов
    и веб-приложений. С помощью популярных фреймворков, таких как Gin, Echo, или
    net/http, можно легко создавать веб-сервисы, API, сайты и другие
    веб-приложения.
    \item Микросервисы: Go является популярным выбором для разработки
    микросервисов благодаря своей производительности, эффективности и поддержке
    конкурентности. Многие крупные компании используют Go для создания своей
    микросервисной архитектуры.
  \end{itemize}
\end{frame}
% Слайд 22 =====================================================================
\begin{frame}[fragile] \frametitle{Недостатки языка Go}
  \begin{itemize}
  \item Ограниченная область применения. Язык больше подходит для сетевых и
  серверных приложений, чем для десктопных. Также он не имеет поддержки для
  создания графических интерфейсов.
  \item Чрезмерная простота синтаксиса. Эта простота — и плюс, и минус. Так как
  некоторые сложные задачи могут потребовать написание большего кода в Go, если
  сравнивать его с другими языками программирования.
  \item Средняя распространенность. Несмотря на рост популярности, Go остается
  нишевым языком. Количество вакансий, где работодатель требует знания Go,
  меньше, чем для других популярных языков программирования по типу Java, Python
  или C++.
  \item Низкий порог входа. Легкость изучения Go приводит к тому, что очень
  многие программисты осваивают его, из-за чего наблюдается рост конкуренции.
  \end{itemize}
\end{frame}
% Слайд 23 =====================================================================
\begin{frame}[fragile] \frametitle{Востребованность}
  Хотя Go и нишевый язык, на рынке иногда наблюдаются всплески его
  популярности. На основе данных из опроса на GitHub, в 2021 году Golang попал в
  ТОП-5 самых востребованных языков и даже опередил C\# и PHP. А в первой
  половине 2023 года Go 10 место в этом же топе.
  \\\\
  Если смотреть глобально и мыслить объективно, Golang все равно востребован на
  рынке. Правда, его востребованность может сильно варьироваться в зависимости
  от региона, отрасли и конкретной компании.
\end{frame}
% Слайд 24 =====================================================================
\begin{frame}[fragile] \frametitle{Популярные проекты на Go}
  \textbf{Docker}. Это одна из самых известных и широко используемых платформ
  для контейнеризации приложений. Основная часть Docker (в том числе Engine)
  написана именно на Go.
  \\\\
  \textbf{Kubernetes}. Оркестратор контейнеров с открытым исходным кодом,
  разработанный для управления и автоматизации контейнеризированных приложений.
  Kubelet и Kubectl — компоненты Kubernetes, написанные на Go.
  \\\\
  \textbf{Etcd}. На Go написано распределенное хранилище ключ-значение в памяти
  Etcd, используемое для хранения конфигураций и данных в Kubernetes.
  \\\\
  \textbf{Prometheus}. Система мониторинга с открытым исходным кодом,
  предназначенная для сбора и анализа метрик приложений и инфраструктуры.
  Основная и серверная часть Prometheus написаны Go-программистами.
\end{frame}
% Слайд 25 =====================================================================
\begin{frame}[fragile] \frametitle{Заключение}
  Go – мощный, безопасный, достаточно простой и очень востребованный язык
  программирования. Он хорошо подходит для создания высокопроизводительных
  систем, поскольку способен повысить производительность программы в 5–10 раз
  без каких-либо оптимизаций. 
  \\\\
  А благодаря гибкости и способности решать разнообразные задачи, использовать
  его можно в самых разных областях – от сетевого программирования до
  криптовалютных приложений.
\end{frame}
% Слайд 26 =====================================================================
% Список литературы
\begin{frame}
  \frametitle{Список использованных источников}
  \small
  \begin{thebibliography}{9}
     \setbeamertemplate{bibliography item}[book]
     \bibitem{go_book}
     {Seguin,~K.} 
     \newblock{The Little Go Book}~/ K.~Seguin.
     \newblock США,  2018. С. 154--174.
     %
     \setbeamertemplate{bibliography item}[online]
     \bibitem{go_url_1} {An Introduction to Programming in Go} 
     \newblock{[{Э}лектронный ресурс]}~/ Go Docs~/ Go Docs.
     \newblock 2024.
     \newblock URL:~{https://www.golang-book.com/}\
       (Дата обращения 25.01.2024). Загл. с экр. Яз. англ.
     %
     \bibitem{go_url_2} {Go in Action} 
     \newblock{[{Э}лектронный ресурс]}~/ Go In Action. 
     \newblock 2024.
     \newblock URL:~{https://www.manning.com/books/go-in-action}\ (Дата
       обращения 22.01.2024). Загл. с экр. Яз. англ.
  \end{thebibliography}
\end{frame}
% Слайд 27 =====================================================================
\begin{frame}
\Huge{\centerline{\textbf{Спасибо за внимание!}}}
\end{frame}

\end{document} 