%-------------------------------------------------------------------------------
%  PACKAGES AND THEMES
%-------------------------------------------------------------------------------
\documentclass{beamer}
\mode<presentation> {
  %-----------------------------------------------------------------------------
  % Назначение всех переменных. 
  % P.S. Для работы с подсветкой кода frame должен быть [fragile] !!!!
  %-----------------------------------------------------------------------------
  \usetheme{Szeged}                                  % Красивая тема
  \usecolortheme[]{beaver}
  \setbeamertemplate{headline}{}                     % Выключить шапку
  \setbeamertemplate{footline}[page number]          % Нумерация страниц
  \setbeamertemplate{navigation symbols}{}           % Выключить стрелки
  \usepackage[T2A]{fontenc}                          % Русский язык
  \usepackage[utf8]{inputenc}                        % И кодировка тоже
  \usepackage[english, russian]{babel}               % Языки. Последний главный.
  \usepackage{graphicx}                              % Включаем картинки
  \usepackage{booktabs}                              % Используем \toprule, 
                                                     % \midrule, \bottomrule
  \usepackage{minted}                                % Подсветка кода
  \usepackage{xcolor}                                % Цвета (расширенная)
  \usemintedstyle{monokai}                           % Белый текст для темы
  \definecolor{dark}{HTML}{282828}                   % Всем темную тему!
  \newcommand{\hlt}[1]{\textcolor{red}{\textbf{#1}}} % Подсветка красным ж.
  \newcommand{\fns}{\footnotesize}

  \newcommand{\imagebox}[6]{
      #1
      \begin{figure}[!ht]
          #2
          \ifx&#3&%
          % #3 is empty
          \includegraphics[width=\textwidth,height=\textheight,
                          keepaspectratio]{#4}
          \else
          \includegraphics[scale=#3]{#4}
          \fi
          \ifx&#5&%
          % #5 is empty
          \label{fig:#6}
          \else
          \caption{#5}\label{fig:#6}
          \fi
      \end{figure}
      #1 
      \phantom{imageboxfix}
  }
}
%-------------------------------------------------------------------------------
%  TITLE PAGE
%-------------------------------------------------------------------------------

\title[Язык Go]{Язык Go. Основные возможности и применение} 
\author{Работа \\ студента 411 группы И.~И.~Королёва \\ 
                  и студента 451 группы Д. ~А. ~Емелина}
\institute{{Саратовский государственный университет} \\ им.~Н.~Г.~Чернышевского
            \\[5pt]
            Кафедра математической кибернетики\\ и компьютерных наук\\[5pt]
            Научный руководитель: доцент Иванова ~А.~С. }
\date{2024}

\begin{document}


%-------------------------------------------------------------------------------
%  СЛАЙДЫ
%-------------------------------------------------------------------------------
% Cлайд 01 =====================================================================
\begin{frame} 
  \titlepage 
\end{frame}
% Слайд 02 =====================================================================
\begin{frame}[fragile] \frametitle{Введение}
  Go "--- относительно молодой, но популярный язык программирования.
  \\\\
  Цель, которая стояла перед создателями Go "--- разработать простой и
  эффективный язык программирования, который мог бы использоваться для
  создания качественного программного обеспечения.
  \\\\
  Golang получил третье место в рейтинге языков программирования,
  которые хотели бы освоить разработчики.
\end{frame}
% Слайд 03 =====================================================================
\begin{frame}[fragile] \frametitle{Мотивация}
  \begin{itemize}
    \item Первый процессор Pentium 4 с тактовой частотой 3,7 ГГц был представлен
    еще в 2005 году корпорацией Intel. 
    \item Macbook Pro 2023 имеет тактовую частоту 4 ГГц. 
  \end{itemize}
  Итог "--- почти за два десятилетия мощности не слишком-то изменились.\\

  Для увеличения производительности решили:
  \begin{itemize}
    \item добавлять в процессоры все больше и больше ядер;
    \item ввести гиперпоточность (концепция параллельного выполнения задач);
    \item добавить больше кэша.
  \end{itemize}
\end{frame}
% Слайд 04 =====================================================================
\begin{frame}[fragile] \frametitle{Мотивация}
  Если же нельзя положиться на усовершенствование оборудования, единственный
  выход "--- более эффективное ПО для повышения производительности. 
  \\\\
  По итогу выходит, что разрабатываемое нами ПО и языки программирования должны
  поддерживать параллелизм и быть расширяемыми в условиях постоянного увеличения
  количества ядер. 
  \\\\
  Большинство современных языков программирования (таких как Java, Python)
  поддерживают многопоточность. Но настоящая проблема связана с одновременным
  исполнением, блокировкой потоков, состоянием гонки и взаимоблокировками.
\end{frame}
%-------------------------------------------------------------------------------
% Слайд 05 =====================================================================
\begin{frame}[fragile] \frametitle{Мотивация}
  Возьмем, к примеру, Java. 
  \\\\
  Каждый канал потребляет около 1 Мб объема памяти, и, в конце концов, если вы
  задействуете тысячи потоков, все может закончиться нехваткой памяти. Кроме
  того, взаимодействие между двумя или несколькими потоками – это тоже непросто.
  \\\\
  Язык Go (он же Golang) появился в 2009 году, когда уже были многоядерные
  процессоры. Это позволило не реализовывать параллельные механики поверх уже
  имеющейся спецификации языка, а разработать язык сразу с учетом параллелизма. 
\end{frame}
%-------------------------------------------------------------------------------
% Слайд 06 =====================================================================
\begin{frame}[fragile] \frametitle{Преимущества языка. Простота}
  \textbf{1. Простота} 
  \\\\
  У Go достаточно простой синтаксис (с определенными допущениями), поэтому
  приложения можно разрабатывать быстрее, чем на некоторых других языках.
  \\\\
  Golang достаточно быстро может изучить как полный новичок в
  программировании, так и уже «сформировавшийся программист», тот, кто уже знает
  один или несколько языков.
\end{frame}
%-------------------------------------------------------------------------------
% Слайд 07 =====================================================================
\begin{frame}[fragile] \frametitle{Преимущества языка. Простота}
  Go специально не учитывает многие особенности современных языков ООП в угоду
  простоте кода и удобству его поддержки:
  \begin{enumerate}
    \item Нет классов. Все разделяется на пакеты. Go работает со структурами, а
    не с классами.
    \item Не поддерживает наследование. Это упрощает изменение кода. Без
    наследования Go также становится легко читаемым: нет суперклассов, которые
    следует изучать особо тщательно.
    \item Нет аннотаций (метаданных, прикрепляемых к коду для различных целей).
    \item Нет конструкторов.
    \item Нет исключений.
  \end{enumerate}
\end{frame}
%-------------------------------------------------------------------------------
% Слайд 08 =====================================================================
\begin{frame}[fragile] \frametitle{Преимущества языка. Простота}
  Вышеизложенные изменения делают Golang отличным от других языком, а
  программирование на Go – предельно простым. 
  \\\\
  В отличие от прочих новых языков, таких как Swift, синтаксис Go стабилен. Он
  остался прежним с первого выпуска версии 1.0, что состоялся в 2012 году. Это
  делает его обратно совместимым.
  
\end{frame}
% Слайд 09 =====================================================================
\begin{frame}[fragile] \frametitle{Преимущества языка. Горутины}
  \textbf{2. Горутины}
  \\\\
  У Go есть goroutine вместо потоков. Они потребляют только 2 Кб памяти. Таким
  образом, можно в любой момент активировать миллионы горутин.
  \imagebox{}{\centering}{}{resources/01-goroutines.jpg}{}{pic01}
\end{frame}
% Слайд 10 =====================================================================
\begin{frame}[fragile] \frametitle{Преимущества языка. Горутины}
  Другие преимущества:
  \begin{enumerate}
    \item Горутины используют больше памяти только тогда, когда это необходимо.
    \item Также они запускаются быстрее, чем потоки.
    \item Более того, они идут вместе со встроенными примитивами, чтобы
    безопасно обмениваться данными.
    \item В условиях одновременного использовании структур данных не придется
    прибегать к блокировке мьютексов.
    \item 1 горутина может свободно работать на нескольких потоках. Горутины
    мультиплексируются в небольшое количество потоков ОС.
  \end{enumerate}
\end{frame}
% Слайд 11 =====================================================================
\begin{frame}[fragile] \frametitle{Преимущества языка. Работа с памятью}
  \textbf{3. Работа с памятью и компиляция}
  \\\\
  Одним из наиболее значительных преимуществ языков C и C++ над другими
  современными языками, такими как Java/Python, является их производительность.
  Дело в том, что C и C++ вместо интерпретации компилируются.
  \\\\
  Как и языки низкого уровня, Go является компилируемым. Это означает, что
  производительность почти такая же высокая, как и в низкоуровневых языках. А
  еще он использует сборщик мусора для выделения и удаления объекта.
  
\end{frame}
% Слайд 12 =====================================================================
\begin{frame}[fragile] \frametitle{Преимущества языка. Чистота кода, типизация}
  \textbf{4. Чистота кода и статическая типизация}
  \\\\
  Компилятор Go позволяет держать код «чистым». К примеру, неиспользуемые
  переменные считаются ошибкой компиляции. В Go решается большая часть проблем
  форматирования. Это делается, к примеру, при помощи программы gofmt при
  сохранении или компиляции. Форматирование правится автоматически.
  \\\\
  Язык программирования Go имеет статическую, строгую типизацию. Это означает,
  что типы данных определяются на этапе компиляции и проверяются на соответствие
  во время выполнения программы.
\end{frame}
% Слайд 13 =====================================================================
\begin{frame}[fragile] \frametitle{Преимущества языка. Не нужны фреймворки}
  \textbf{5. Не нужны фреймворки}
  \\\\
  В Go отсутствует традиционное понятие фреймворков, которые часто используются
  в других языках программирования по типу Python, Ruby, JavaScript и т. д.
  Вместо этого в Гоу для создания приложений применяются модули и библиотеки.
  \\\\ 
  Разработчики используют библиотеки и инструменты, чтобы строить мощные и
  надежные Go-приложения, на 100\% подходящие для решения задач, поставленных
  перед ними.
\end{frame}
% Слайд 14 =====================================================================
\begin{frame}[fragile] \frametitle{Преимущества языка. Не нужны фреймворки}
  \textbf{Почему отсутствие фреймворков — это плюс для разработчика?}
  \begin{itemize}
    \item Гибкость.
    \\
    Вместо принудительного использования фреймворков, Go предоставляет
    разработчикам свободу выбора библиотек и инструментов в зависимости от
    каждого проекта. Это позволяет строить приложения, оптимизированные под
    решение конкретных задач.
    \item Модульность. 
    \\
    Программирование на Go — модульный подход к разработке. Этот подход
    позволяет импортировать сторонние библиотеки и использовать только те
    функции, которые действительно нужны в процессе разработчик. То есть
    модульность минимизирует зависимости. 
  \end{itemize}
\end{frame}
% Слайд 15 =====================================================================
\begin{frame}[fragile] \frametitle{Преимущества языка. Не нужны фреймворки}
  \textbf{Почему отсутствие фреймворков — это плюс для разработчика?}
  \begin{itemize}
    \item Множество библиотек.
    \\
    Сообщество Golang активно разрабатывает и поддерживает много полезных
    библиотек и инструментов, которые можно использовать в проектах без
    необходимости использования фреймворков. Эти библиотеки позволяют
    реализовать функциональные возможности по типу маршрутизации HTTP, работы с
    БД, обработки форм и т. п.
    
    \item Производительность. 
    \\
    Go изначально разрабатывался с акцентом на высокую скорость работы.
    Отсутствие больших и тяжеловесных фреймворков способствует более низкому
    потреблению ресурсов рабочей машины и обеспечивает более быстрый запуск
    приложений.
  \end{itemize}
\end{frame}
% Слайд 16 =====================================================================
\begin{frame}[fragile] \frametitle{Преимущества языка. Маппинги}
  \textbf{6. Маппинги}
  \\\\
  Маппинги (Maps) в языке программирования Go представляют собой встроенную
  структуру данных, которая представляет собой набор пар ключ-значение, где
  каждый ключ уникален в рамках маппинга. Они известны также как ассоциативные
  массивы или словари в других языках программирования.
\end{frame}
% Слайд 17 =====================================================================
\begin{frame}[fragile] \frametitle{Преимущества языка. Маппинги}
  \textbf{Основные характеристики маппингов в Go}
  \\\\
  \begin{enumerate}
    \item Уникальные ключи: Каждый ключ в маппинге уникален, что означает, что в
    маппинге не может быть двух элементов с одинаковыми ключами.
    \item Неупорядоченность: Элементы в маппинге не хранятся в определенном
    порядке. Это означает, что порядок элементов при обходе маппинга не
    гарантирован и может меняться между разными запусками программы.
    \item Динамическое изменение размера: Маппинги в Go могут динамически
    увеличиваться или уменьшаться по мере добавления или удаления элементов.
  \end{enumerate}
\end{frame}
% Слайд 18 =====================================================================
\begin{frame}[fragile] \frametitle{Что можно написать на Go?}

\end{frame}


\end{document} 